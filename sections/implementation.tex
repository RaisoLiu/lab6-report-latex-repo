
\section{Implementation}


\subsection{Denoising Diffusion Probabilistic Model}

Denoising Diffusion Probabilistic Model 也稱作 DDPM,是一種利用不斷地加入噪音到圖片中,並且學習如何去噪的模型,加入噪音的步驟叫做前向過程,去噪的步驟叫做反向過程,而這個過程中會經過許多的步驟,每個步驟都會加入一點點的噪音,最後會得到一個非常模糊的圖片,然後模型會學習如何從這個模糊的圖片中,一步一步地恢復成原本的圖片,由於我們可以掌握每一步驟加入的造音是什麼,所以有 groundtruth 可以學習,所以這個過程可以被訓練成一個去噪的過程。

實際訓練步驟如下:

\begin{itemize}
    \item 從所有圖片中抽取一張圖片,稱作 $x_0$
    \item 然後從 1 \~ t 中選擇一個數字,稱作 t,這裡我設定 t 為 1000 步。
    \item 然後生成一個噪音的圖片,接著計算真實圖片與噪音圖片,在給的步數 t 時,兩者的混合比例。
    \item 接著把 noise 的圖片與 t 與 condiaion embedding 一起送入模型,模型會預測這個圖片在 t 時的 noise 是什麼。
    \item 由於我想要準確的預測噪音,所以這裡是一餓回歸任務,回歸每個圖片的每個座標的噪音數值,所以我使用 MSE loss 來協助模型學習。
    \item 接著計算梯度,然後更新模型。
    \item 重複上述策略直到 epoch 設定數量結束。
\end{itemize}












\subsection{Model Architecture}

由於需求是按照給定的 label 來生成圖片,因此要考慮 label 的資訊如何融入模型中。

這裡我使用 diffusers ,來自 Hugging Face 的 library 創建 conditionalDDPM,主要架構使用一個 U-Net 的架構,首先我會把 conditional 的資訊透過一層 fc 編碼成 embedding ,以及把 step 的 t 編碼成一個 embedding,並且把這兩個相加,接下來 diffusers 建構的 Model 中,每個 layer 會有兩個輸入,一個是來自上一層的 UNet 的輸出,另一個則是來自於我的 embedding,我設計的 embedding 輸入到每個 layer 之時,每個 layer 會在準備一個 fc 層,把 embedding 轉換成適合的維度,接著我考慮從上一個 layer 來自的輸入,會先經過主架構,也就是 U-Net 的 CNN,然後接著把這兩個輸入相加,當作這個 layer 的輸出。

接著考慮一個案例,就是一個貓的尾巴與頭在圖片上的距離可能會很遠,但其實彼此是有關聯的,所以這裡我在比較高層次的 layer 中,加入一個 attention 的機制,來讓模型學習到這個關聯性,因為比較高層次的 layer 被相信帶有比較多的語意訊息,所以這裡我在比較高層次的 layer 中,加入一個 attention 的機制,來讓模型學習捕捉全局特徵和上下文。

那為什麼不在所有的 layer 都加入呢? 因為淺層的網路需要先專注在 local 的特徵上面,先建立對於 local 的理解,所以 attention 的機制可能只是浪費計算資源,並不會帶來多少效果,所以只在比較抽象語意的地方加入。




\subsection{Noise Schedule}
雖然我可以不斷地加入噪音,但是對於資源的調度來說非常吃力,為了計算第 1000 步,我需要把前 999 步的資訊都保留,這樣的計算量非常大,所以這裡我使用 squaredcos\_cap\_v2 來計算如果從最純淨的圖片開始,經過 t 步後,會變成什麼樣子,他跟噪音的相互權重是多少比例,這樣就可以減少計算量,並且保留足夠的資訊。

在原始的論文中,是使用線性的方式設計噪音與真實圖片的混合比例,但是接下來的研究發現,噪音的曲線對於結果的學習很重要,想像一個圖片他的訊號包含高頻與低頻,如果每一個步驟都加入一樣程度的噪音,高頻細節訊號很快就被噪音訊號蓋住,使得模型尚未對高頻細節做充分的理解,所以這裡使用 squaredcos 的曲線來設計噪音與真實圖片的混合比例,這樣可以讓模型在學習的過程中,慢慢加入噪音,讓模型有更多時間,好好的學習細節。











\subsection{Classifier Guidance or no?}

這裡助教有提供一個訓練好的分類器,讓同學可以判斷訓練出來模型的表現,另外同學也可以選擇使用這個分類器來當作引導,讓模型可以有執導的訊號,更快的收斂,但是這會讓模型有分類器的偏見,所以我的信念裡,是希望模型可以自己學到圖片的架構,而不是像是填鴨式教育,所以這裡我選擇不使用分類器。




