\section{Introduction}


這次的目標是學會使用 conditional 的資訊來生成圖片,透過降噪的過程,學會如何從一個模糊的圖片,一步一步的恢復成一個清晰的圖片,並且學會如何使用 conditional 的資訊,來生成我想要的圖片。

這篇作業我將介紹所採用的 Conditional DDPM 模型。在實作方面,我使用了 Hugging Face 的 diffusers 資料庫來建構 U-Net 模型,並整合了條件資訊與時間步長(time step)的嵌入(embedding)。為了捕捉圖像中的遠距離依賴關係,我在模型的較高層次中加入了注意力機制。此外,我選擇了 squaredcos 雜訊排程(noise schedule)以期改善模型對細節的學習,並且決定不使用分類器引導(classifier guidance)以鼓勵模型自主學習圖像結構資訊。

在結果與討論部分,我將展示模型生成的合成圖像、不同訓練階段的去噪過程可視化結果,以及模型在測試集上的分類準確率。我也進行了關於不同批次大小(batch size)對訓練影響的額外實驗,並比較了在固定模型更新次數下,batch size 16 與 160 的訓練情況,為此我實作了自動混合精度(AMP)訓練以在有限的硬體資源下訓練較大的 batch size。這些實驗結果將幫助我評估模型的性能,並討論實作過程中的觀察與發現。

